\documentclass[10pt]{article} 

\usepackage{fullpage}
\usepackage{bookmark}
\usepackage{amsmath}
\usepackage{amssymb}
\usepackage[dvipsnames]{xcolor}
\usepackage{hyperref} % for the URL
\usepackage[shortlabels]{enumitem}
\usepackage{mathtools}
\usepackage[most]{tcolorbox}
\usepackage[amsmath,standard,thmmarks]{ntheorem} 
\usepackage{physics}
\usepackage{pst-tree} % for the trees
\usepackage{verbatim} % for comments, for version control
\usepackage{tabu}
\usepackage{tikz}
\usepackage{float}
% sets
\newcommand{\complex}{{\mathbb C}}
\newcommand{\reals}{{\mathbb R}}
\newcommand{\rationals}{{\mathbb Q}}
\newcommand{\integers}{{\mathbb Z}}

% Math notation
\newcommand{\trnorm}[1]{\left\| #1 \right\|_{\mathrm{tr}}}
\newcommand{\size}[1]{\left| #1 \right|}
\newcommand{\set}[1]{\left\{ #1 \right\}}
\newcommand{\floor}[1]{{\lfloor #1 \rfloor}}
\newcommand{\ceil}[1]{{\lceil #1 \rceil}}

\DeclareMathOperator{\fidelity}{F}
\DeclareMathOperator{\diag}{diag}
\DeclareMathOperator{\support}{supp}

\DeclareMathOperator{\Order}{O}
\DeclareMathOperator{\Span}{span}
\DeclareMathOperator{\expct}{{\mathbb E}}

\newcommand{\transpose}{\top}
\newcommand{\e}{{\mathrm{e}}}
\newcommand{\complexi}{{\mathrm{i}}}

\newcommand{\id}{{\mathbb 1}}
\newcommand{\ind}{{\mathbb 1}}

\newcommand{\poly}{{\mathrm{poly}}}

% linear algebra, geometry, QI
\newcommand{\linear}{{\mathsf L}}
\newcommand{\unitary}{{\mathsf U}}
\newcommand{\sphere}{\mathsf{Sphere}}
\newcommand{\qstate}{{\mathsf D}}
\newcommand{\Pos}{{\mathsf{Pos}}}
\newcommand{\su}[2]{\sum_{#1}^{#2}}

% quantum physics notation
\newcommand{\density}[1]{\ketbra{#1}{#1}}
\newcommand{\ancilla}{\ket{\bar{0}}}

% mnemonics
\newcommand{\eqdef}{\coloneqq}
\newcommand{\tensor}{\otimes}
\newcommand{\Tensor}{\bigotimes}
\newcommand{\xor}{\oplus}
\newcommand{\union}{\cup}
\newcommand{\intersect}{\cap}
\newcommand{\meet}{\wedge}
\newcommand{\up}{\uparrow}
\newcommand{\down}{\downarrow}
\newcommand{\compose}{\circ}
\newcommand{\adjoint}{*}


\lstnewenvironment{python}{
\lstset{frame=tb,
language=Python,
aboveskip=3mm,
belowskip=3mm,
showstringspaces=false,
columns=flexible,
basicstyle={\small\ttfamily},
numbers=none,
numberstyle=\tiny\color{Green},
keywordstyle=\color{Violet},
commentstyle=\color{Gray},
stringstyle=\color{Brown},
breaklines=true,
breakatwhitespace=true,
tabsize=2}
}
{}

\lstnewenvironment{cpp}{
\lstset{
backgroundcolor=\color{white!90!NavyBlue},   % choose the background color; you must add \usepackage{color} or \usepackage{xcolor}; should come as last argument
basicstyle={\scriptsize\ttfamily},        % the size of the fonts that are used for the code
breakatwhitespace=false,         % sets if automatic breaks should only happen at whitespace
breaklines=true,                 % sets automatic line breaking
captionpos=b,                    % sets the caption-position to bottom
commentstyle=\color{Gray},    % comment style
deletekeywords={...},            % if you want to delete keywords from the given language
escapeinside={\%*}{*)},          % if you want to add LaTeX within your code
extendedchars=true,              % lets you use non-ASCII characters; for 8-bits encodings only, does not work with UTF-8
% firstnumber=1000,                % start line enumeration with line 1000
frame=single,	                   % adds a frame around the code
keepspaces=true,                 % keeps spaces in text, useful for keeping indentation of code (possibly needs columns=flexible)
keywordstyle=\color{Cyan},       % keyword style
language=c++,                 % the language of the code
morekeywords={*,...},            % if you want to add more keywords to the set
% numbers=left,                    % where to put the line-numbers; possible values are (none, left, right)
% numbersep=5pt,                   % how far the line-numbers are from the code
% numberstyle=\tiny\color{Green}, % the style that is used for the line-numbers
rulecolor=\color{black},         % if not set, the frame-color may be changed on line-breaks within not-black text (e.g. comments (green here))
showspaces=false,                % show spaces everywhere adding particular underscores; it overrides 'showstringspaces'
showstringspaces=false,          % underline spaces within strings only
showtabs=false,                  % show tabs within strings adding particular underscores
stepnumber=2,                    % the step between two line-numbers. If it's 1, each line will be numbered
stringstyle=\color{GoldenRod},     % string literal style
tabsize=2,	                   % sets default tabsize to 2 spaces
title=\lstname}                   % show the filename of files included with \lstinputlisting; also try caption instead of title
}
{}

% floor, ceiling, set
\DeclarePairedDelimiter{\iprod}{\langle}{\rangle}

\DeclareMathOperator{\Int}{int}
\DeclareMathOperator{\mean}{mean}

% commonly used sets
\newcommand{\R}{\mathbb{R}}
\newcommand{\N}{\mathbb{N}}
\newcommand{\Q}{\mathbb{Q}}
\renewcommand{\P}{\mathbb{P}}

\newcommand{\sset}{\subseteq}

\theoremstyle{break}
\theorembodyfont{\upshape}

\newtheorem{thm}{Theorem}[subsection]
\tcolorboxenvironment{thm}{
enhanced jigsaw,
colframe=Dandelion,
colback=White!90!Dandelion,
drop fuzzy shadow east,
rightrule=2mm,
sharp corners,
before skip=10pt,after skip=10pt
}

\newtheorem{cor}{Corollary}[thm]
\tcolorboxenvironment{cor}{
boxrule=0pt,
boxsep=0pt,
colback={White!90!RoyalPurple},
enhanced jigsaw,
borderline west={2pt}{0pt}{RoyalPurple},
sharp corners,
before skip=10pt,
after skip=10pt,
breakable
}

\newtheorem{lem}[thm]{Lemma}
\tcolorboxenvironment{lem}{
enhanced jigsaw,
colframe=Red,
colback={White!95!Red},
rightrule=2mm,
sharp corners,
before skip=10pt,after skip=10pt
}

\newtheorem{ex}[thm]{Example}
\tcolorboxenvironment{ex}{% from ntheorem
blanker,left=5mm,
sharp corners,
before skip=10pt,after skip=10pt,
borderline west={2pt}{0pt}{Gray}
}

\newtheorem*{pf}{Proof}
\tcolorboxenvironment{pf}{% from ntheorem
breakable,blanker,left=5mm,
sharp corners,
before skip=10pt,after skip=10pt,
borderline west={2pt}{0pt}{NavyBlue!80!white}
}

\newtheorem{defn}{Definition}[subsection]
\tcolorboxenvironment{defn}{
enhanced jigsaw,
colframe=Cerulean,
colback=White!90!Cerulean,
drop fuzzy shadow east,
rightrule=2mm,
sharp corners,
before skip=10pt,after skip=10pt
}

\newtheorem{prop}[thm]{Proposition}
\tcolorboxenvironment{prop}{
boxrule=0pt,
boxsep=0pt,
colback={White!90!Green},
enhanced jigsaw,
borderline west={2pt}{0pt}{Green},
sharp corners,
before skip=10pt,
after skip=10pt,
breakable
}

\setlength\parindent{0pt}
\setlength{\parskip}{2pt}


\begin{document}
\let\ref\Cref

\title{\bf{cs467}}
\date{\today}
\author{Austin Xia}

\maketitle
\newpage
\tableofcontents
\listoffigures
\listoftables
\newpage
\section{Course Information}
instructor: Ashwin Nayak

Grade: 60\% Assignments+15\%midterm+20\%final
\section{\date{\today}}
\subsection{quantum phenomena}

Experiment:

path 0 denoted by $e_0:=(1,0)\in C^2$

path 1 denoted by $e_1:=(0,1)\in C^2$

a:=b means a is defined to be b

initial state of photon $e_0$, denoted $|\ 0>$(a classcial bit being 0)

beam splitter is a linear transform
$$H:=\frac{1}{\sqrt{2}}\begin{pmatrix}
    1&1\\
    1&-1
\end{pmatrix}$$
hadamard transformation

after hitting the beam splitter, $$state=H|0>=\frac{1}{\sqrt{2}}(1,1)$$
this is called a superposition of state 0 and 1. quantum is simutanously in this state
\newline
\newline
state on passing through 2nd beam splitter
$$state=H\frac{1}{\sqrt{2}}(1,1) = (1,0) = |0>$$
1,0 here are probability amplitude
probablity of state $|0>$ being observed = $|its\ amplitude|^2$
\subsection{qubit, quantum state, measurement}
qubit $\equiv$ quantum bit $\equiv$ register

it is a unit of quantum computation 

a register can be consistered as 1 qubit or a collection of qubit

state of a qubit $\equiv$ described by a unit vector in $C^2$ (Hilbert space)

measurement $\equiv$ experiment

experiment to probing what state the qubit is in, in the simplest case 
it is a \emph{complete projective measurement} 

specified by an orthonormal basis $B:=\{|u_0>, |u_1>\}$

example: 
\begin{itemize}
    \item 
standard basis $\{|0>,|1>\}$, read as capt zero
    \item
Hadamard $\{|+>, |->\}$
    \item 
$\frac{1}{\sqrt{2}}(|0>+i|1>), \frac{1}{\sqrt{2}}(|0>-i|1>)$
\end{itemize}

\begin{defn}
    "bra" v $\equiv$ dual of vector v $\equiv$ conjugate transpose of v
\end{defn}

\begin{ex}
    question: in lecture you write $<0|=(1,0)$, shouldn't it be $bra(|0>)=(1,0)$

    $<0|=(1,0)$

    $<+|=\frac{1}{\sqrt{2}}(1,1)$

    $<v_0| = \frac{1}{\sqrt{2}}(1,-i)$
\end{ex}

the inner product between two vectors $|u>, |v> \in C$

is denoted as $<u|v>:=<u|*|v>$

\begin{ex}
    $<0|+>=\frac{1}{\sqrt{2}}$

    $<+|v_0>=(1+i)/2$, abs value is $\frac{1}{\sqrt{2}}$
\end{ex}

Effect of a measurement in $B:=\{|u_0>, |u_1>\}$
on a qubit in state $|v>$:
\begin{itemize}
    \item we observe outcome "0" or "1"
    \item see outcome $b\in\{0,1\}$, with prob $|<u_b|v>|^2$
    \item when outcome b is observed, state become $|u_0>$,
    the state collapses to $|u_0>$
\end{itemize}

\begin{ex}
    Measure $|+>$ in basis 
    \begin{itemize}
        \item $B:=\{|0>,|1>\}$ outcomes 0/1, when outcome =b, state = $|b>$
        \item $B:=\{|v_0>,|v_1>\}$, outcome 0 with prob $|<v_0|+>|^2 = (\frac{1}{\sqrt{2}})^2=1/2$
        final state = $|v_b>$ when outcome = b
    \end{itemize}
\end{ex}

general case is 
\emph{complete von neumann measurement}

\subsection{bit commitment}
simple, single mesg. protocol

commit stage:
\begin{itemize}
    \item Alice has a bit $a\in \{0,1\}$. She sends a message m (depending on a) to Bob
    \item Bob receives m stores it
\end{itemize}

Reveal phase:
\begin{itemize}
    \item Alice sends bit a, msg r to Bob
    \item Bob use r to check that m is consistent with a, if so accept
\end{itemize}

Requirements:
\begin{itemize}
    \item Hiding property: Bob cannot learn bit 'a' from message m 
    \item Binding property: Alice cannot send bit $\bar{a}$, and some message r s.t. Bob accept
\end{itemize}

Classically, Bob can learn bit a from m or Alice can cheat

Quantumly, we can have a protocal that solve this 

\emph{Quantum Protocal}
\begin{itemize}
    \item 
Alice has a state $a\in \{0,1\}$ Perpares a qubit M in state $|\phi_a>$
    \item She sends M to Bob. who stores this qubit, this complete the commitment stage
\end{itemize}

$\theta := \pi/8 \\ |\psi_0> := \alpha |0> + \beta |1>\\
\alpha := cos(\pi/8), \beta = sin(\pi/8)\\|\psi_1>:=\beta |0> + \alpha |1>$

Reveal Stage 
\begin{itemize}
    \item Alice send bit a, and no other msg r 
    \item Bob measures qubit M in basis
     $\{|\psi_a>, |\tilde{\psi_a}>\}$

     He accepts (a) if cutcome = 0, reject o.w.
\end{itemize}

Proposition: this protocal satisfies the hiding and binding properties
in a probablitistic sense:

(a) Given the qubit M, regardless of which measurement Bob makes,
P(outcome=a)$\leq \delta$ for some $\delta < 1 $

(b) For any state in which Alice prepares M, if she wishes to claim bit b 
in the reveal stage, 
P(Bob accepts b)$\leq \delta$ for some $\delta < 1 $

Hiding property:

we can show that the optimal measurement $\{|u_0>,|u_1>\}$
s.t. Pr(outcome=a | $|\psi_a>$) is maximized is given by 
$|u_0> := |0>, |u_1>:=|1>$

Concealing property 

claim: $|\psi>$ is state that maximizes the $$min_{a\in \{
    0,1\}} |<\psi_a|\psi>|^2$$

\subsection{multiple qubit}
general quantum state is a unit vector in $\mathbf{C}^d$
, $d:=2^n$ spanned by $|x\rangle (e_x)$
$|\psi\rangle :=\sum_x\alpha_x|x\rangle$
unit vector means $\sum_{x\in\{0,1\}^n}\abs{\alpha_x}^2=1$

suppose $|u\rangle \in C^{d_1}$, $|v\rangle \in C^{d2}$ the tensor product 
$|u\rangle \otimes |v\rangle$ is vector in $C^{d1*d2}$

suppose we have indexed unit vectors $\{e_i\}, \{f_j\}$ are std basic vectors 
for $C^{d1}, C^{d2}$ 
$$g_{i,j}:=(0,0,0...1,0,0,0...)~~ \text{i is the } (i-1)d_1+j$$

suppose 
$|u\rangle=\sum u_ie_i\in C^{d1}$
$|v\rangle=\sum v_if_i\in C^{d2}$

$$|u\rangle \otimes |v\rangle = \sum_{ij}u_iv_jg_{ij}$$

in dirac notation, it is $$\sum_{ij}u_iv_j|i,j\rangle$$

note $|u\rangle|v\rangle = |u,v\rangle=|uv\rangle$

tensor product is bilinear

$$|\psi_1\rangle$$

$$C^{d1d2}\neq\{|u\rangle\otimes|v\rangle: |u\rangle\in C^d 
|v\rangle \in C^{d2}\}$$

however
$$C^{d1d2}=span\{|u\rangle\otimes|v\rangle: |u\rangle\in C^d 
|v\rangle \in C^{d2}\}$$

this span is $C^{d1}\otimes C^{d2}$

product state if can be written as $|a\rangle|b\rangle$

entangled state if cannot 

\subsubsection{properties of tensor products of operator}
\begin{itemize}
    \item $(\alpha U)\otimes V=\alpha(U\otimes V)= U\otimes(\alpha V)$
    \item $(U_1+U_2)\otimes V=U_1 \otimes V+U_2\otimes V$
    \item $U\otimes (V_1+V_2=U\otimes V_1+U\otimes V_2)$
    \item $(U_1\otimes V_1)(U_2 \otimes V_2)=(U_1U_2)\otimes(V_1V_2)$
    \item $(U\otimes V)^*=U^* \otimes V^*$
    \item $(U\otimes V)^{-1}=U^{-1}\otimes V^{-1}$
    \item $\norm{U\otimes V} = \norm{U}*\norm{V}$
\end{itemize}
\subsubsection{inner product}
suppose $|u_1v_1\rangle, |u_2v_2\rangle \in C^{d1} \otimes C^{d2}$
their inner product=$$(\langle u_1v_1|)(|u_2v_2\rangle)=\langle U_1u_2\rangle\langle v_1v_2\rangle$$

suppose $w, w_1, w_2\in C^{d1} \otimes C^{d2}$

$$E_{ij}=e_ie_j^T=e_ie_j^*=|i\rangle\langle j|\\
U\sum\alpha_{ij}|i\rangle\langle j|$$
\subsection{measurement}
measurement of multiple qubits 

A sequence of qubit (registe) M in state $|\psi\rangle\in C^d$

A complete projective measurement of M is specified by an orthonormal basis
$$B=\{|u_i\rangle: i\in[d]\}$$

\subsubsection{transmission of polorized light}

photon pass through poloarizing film, with a detector on the other side,

\begin{itemize}
    \item a photon is either absorbed or passes through
    \item only $|\rightarrow\rangle$ component pass through 
    \item if state is $\alpha|\rightarrow\rangle+\beta|\uparrow\rangle$
    we observe the photon at D with probability $|\alpha|^2$
    \item if we place a second film fim also oriented horizentally, all light is transmitted
\end{itemize}



coarser measurment: projective measurement 

specified by a sequence of orthogonal projection operation
$\{P_i:i\in[k], \sum^k_{i=1}=I\}$

On measurement of M in state $|\psi\rangle\in C^d$
\begin{itemize}
    \item we observe a probablitistic coutcome $i\in [k]$
    \item $p(outcome=i|...)=\norm{P_i|\psi\rangle}^2$
    \item on outcome i, state of M becomes $$\frac{P_i|\psi\rangle }{\norm{P_i|\psi\rangle}}$$
\end{itemize}

\subsubsection{general measurement in an 0-n basis}
$B:=\{\ket{u_i}\}$\\$P_i:=\ketbra{u_i}{u_i}$\\$P_i$ is orthogonal proj

the effect of measuring $\ket{\varphi}$ in B or in $\{P_i\}$ is the same

\begin{ex}
    we want to check if 2 bit of a 2-qubit state is the same 

    we apply projection $$p_0=\ketbra{00}{00}+\ketbra{11}{11}$$$$p_1=\ketbra{01}{10}+\ketbra{10}{01}$$
    
    when we measure $\ket{+-}$ we have 

    $$pob=\norm{p_0\ket{+-}}^2=\norm{\ketbra{00}{00}\ket{+-}+
    \ketbra{11}{11}\ket{+-}}=\abs{\braket{00}{+-}}^2+\abs{\braket{11}{+-}}^2=1/2$$

    state becomes $\frac{p_0\ket{+-}}{1/\sqrt{2}}=\frac{1}{\sqrt{2}}(\ket{00}-\ket{11})$
\end{ex}

\subsubsection{measuring subsystem}
we've got register AB, both have several bits, 
with state space $C^{d1}\otimes C^{d2}$
say state of AB is $\ket{\psi}$ we wish to measure register A with projective measurement

measurement $\{P_i:i\in[k]\}$ This is equivalent to measuring AB according to 
$$\{P_i\otimes 1: i\in[k]\}$$

we can express $\ket{\psi}$ as  $\ket{\psi}=\su{i=1}{d_1}\alpha_i\ket{u_i\psi_i}$
where $\ket{\psi_i}\in\complex^{d_2}, \norm{\ket{\psi_i}}=1$, but $\{\ket{\psi_i}\}$
need not be orthogonal

question:how

remark on measurement:
$$\ket{\psi}=\su{i=1}{d}p_i\ket{\psi}$$ 
$$1=\norm{\ket{\psi}}^2=\su{i=1}{d}\norm{p_i\ket{\psi}}^2$$

\subsubsection{general measurement}
we will call it a measurement

A general measurement of a resgister M consist of:

prepare another register $M'$ in a fixed state, $\ket{\bar 0}$

measure $MM'$ with a projective measurement on $C^{d1}\otimes C^{d2}$
where M has state space in $C^{d1}$, M' in $C^{d2}$

information content of n qubit:

n-qubit state $\ket{\psi}:=\su{x\in\{0,1\}^n}{}\alpha_x\ket{x}$

discription: $2^n$ parameters

However, we may only reliably encode $\theta(n)$ bits into n qubits

\begin{thm}
    let $x\in\{0,1\}^m$ be uniformly random. suppose we encode $x\in\{0,1\}^m$
    by n-qubit state $\ket{\psi_x}$ Let Y be outcome of any measurement of the state $\psi_x$

    Let y be outcome of any measurement of state $\ket{\psi_x}$ then $$pr(Y=X)\leq 2^n/2^m$$

    \begin{pf}
        suppose we measure state according to $\{P_y: y\in\{0,1\}^m\}$
        ,where outcome y indicates our guess for the encoded string.

        Given state $\ket{\psi_x}$ we append $\ket{\bar 0}\in \complex^{d1}$ and measure 
        according to proj measurement 

        Note $\sum P_y=1$ then $\exists$ basis $\{\ket{f_{yi}}\}$ o.n. 
        $$P_y=\sum_i\ketbra{f_{yi}}{f_{yi}}$$

        $pr(y=x)=\frac{1}{2^m}\sum_xPr(y=x|\ket{\psi_x})$
    \end{pf}
\end{thm}

evolution of quantum bit 
\begin{itemize}
    \item linear 
    \item reversible/invertible
    \item norm preserving
\end{itemize}

computation with qubit may be implemented by allowing system to involve or for subsets of qubit to evolve 
while maintaining the state of the rest.

if subregister A of register AB evolves according to operator U, evolution of AB is given by 
$U\otimes 1$ Note $U\otimes 1$ unitary $\leftrightarrow$ U unitary 

a sophiscated computation may involve sequence of such unitary operators applied to different subregisters 

all operations allowed by laws of quantum physics can be expressed as a composition of 
\begin{itemize}
    \item addition of ancilla
    \item unitary evolution of the entire system 
    \item a projective measurement
\end{itemize}

\subsection{superdense coding}
if A and B hold $E_1, E_2$ respectively, where join state is 
$\frac{1}{\sqrt{2}}(\ket{00}+\ket{11})$

suppose alice has 2 bits. she can apply unitary $U_{ab}=X^aZ^b$ to them 

question can we create entangled state

question cheat by entangled

\subsection{teleportation}
A B connected by classical channel, can send classical bit 

A given qubit M in state $\alpha\ket{0}+\beta\ket{1}$
She would help B construct $\ket{\psi}$

\begin{thm}
    there is a protocal if Alice and Bob share $E_1, E_2$

    two qbit in state $\ket{\psi_{00}}=\frac{1}{\sqrt{2}}(\ket{00}+\ket{11})$


    the protocal:

    alice has M $E_1$, Bob has $E_2$.

    M in $\alpha\ket{0}+\beta\ket{1}$ Bob has $E_2, E_1E_2$ in state $\ket{\psi_{00}}$

    Alice measures qubits $ME_1$ in Bell basis $\{\ket{\psi_{ab}}\}$
    she sends the 2bit outcome to Bob

    Bob recieves $ab\in\{0,1\}$ and applies $U_{ab}$ on $E_2$

    $U_{ab}=X^aZ^b$

    where xz are the standard pauli operation

    claim: the final satate of $E_2$ is $\ket{\psi}$
\end{thm}

An algorithm/program/cuicuit:

$f\{0,1\}^n\rightarrow\{0,1\}^m$
\begin{itemize}
    \item number of bits in the memory \emph{size}, holding input and workspace
    \item a string of s bits, to which memory is initialized. the first n to $\{0,1\}^n$ the rest 0
    \item a sequence of logic gate from a fixed set G
    \item the index of register that represent the output
\end{itemize}

\section{circuit}
    time: number of gates 

    space: number of wire segment
    \\\\
    correctness of a random circuit:

    let $f:\{0,1\}^n \rightarrow \{0,1\}^m$. we say C computes f is $Pr(C(x)=f(x))\geq 2/3$
    for all inputs x

    \subsection{quantum circuit}
    \begin{itemize}
        \item memory consist of qubits 
        \item quantum gates, performing unitary operation 
        \item measurement
    \end{itemize} 
    it is useful to write quantum gate as CNOT=$\ketbra{0}{0}\otimes I+\ketbra{1}{1}\otimes X$

    we can prepare bell state $\ket{\phi_{00}}=\frac{1}{\sqrt{2}}(\ket{00}+\ket{11})$
    by:$$\ket{\phi_{00}}=CNOT^{E_2E_1}(1\otimes H)\ket{00}^{E_1E_2}$$ 

    Note: we use register on which oeprator act AND the register which are in that state 

    $Z=HXH~~~Y=iXZ~~~HYH=i(HXH)(HZH)=iZX=-Y$

    measurement:

    we can perform a basis change to unitary, mesaure in unitary, than change it back.


    \subsection{universality}
        quantum physics, any unitary operator can occur,

        can circuit do it?

        we say circuit computes U on n qubits $C^{2^n}$, if for every state 
        $\ket{\varPsi}\in C^d$, final state is $U\ket{\varPsi}\ket{0}$ with probability 1

        \begin{thm}
            for any unitary operation U, there is a quantum circuit that uses 
            only CNOT and single qubit gate and compute U
        \end{thm}

        but how single qubit gate? how irrational number?

        can we use small number of gate? can we have not precise gate?

        \begin{defn}
            V approximate U if $\norm{U-V}\leq\epsilon$

            this suffice as $\norm{V\ket{\phi}-U\ket{\phi}}\leq \norm{V-U}$

            when output is close, the measurement statistics also close, 
            closeness of probability is measured in l1 distance
            $$\norm{p-q}_1:=\su{k}{i=1}\abs{p_i-q_i}$$
        \end{defn}

        proposition: when $\norm{\ket{\psi}-\ket{\phi}}\leq \epsilon$
        $$\norm{p-q}_1\leq 2\epsilon$$

        proof:

        suppose we measure according to $\{P_i: i\in[k]\}$

        $p_i:=\norm{P_i\ket{\psi}}^2~~~~q_i:=\norm{P_i\ket{\psi}}^2$
        $$\sum \abs{ab}=(\sum a^2)^{1/2}(\sum b^2 )^{1/2}~~~Cauchy-Schwarg$$
        $$\abs*{\norm{a}-\norm{b}}^2\leq\norm{a-b}^2$$
    \subsection{approximating unitary operations}
        if C computes f with probability 2/3 then if $\norm{U-V}\leq \epsilon$,
        then probability $\bar C$ compute f with probability $\geq 2/3-2\epsilon$
        
        proposision:
        if pq are probability distribution on [k], E be any event;
        $$\abs*{p(E)-q(E)}\leq \frac{1}{2}\norm{q}_1$$

    \subsection{universality}
        if $\norm{U-V}\leq \epsilon$
        output state are within $\epsilon$ in Euclidean distance,
        distributions over measurement are within $2\epsilon$ in $l_1$ distance,

        \begin{thm}
            For any unitary operation U on $\complex^d$,
            $\epsilon \in (0,1)$ there is a quantum circuit that uses only 
            gates from $\{CNOT, H, T\}$  and computes a unitary operation U

            we say the gate set is universal
        \end{thm}


        what is overhead of approximating unitary operation using these 2 gates?

        \begin{thm}
            for any $\epsilon$ and operation $U\in U(2)$ there is 
            a sequence of $O(log\frac{1}{\epsilon})^c$ gates from $\{H, T\}$
            which computes V where c is a universal constant
        \end{thm}

        how does the error scale?

        Proposition: if $\norm{U_i-V_i}\leq \epsilon$, then 
        $$\norm{V_t...V_2V_1-U_t...U_2U_1}\leq t\epsilon$$

        to get overall error $\epsilon$ when approximating m gates, we need 
        only approximate each gate with error $\epsilon/m$, with a sequence of 
        $O(log\frac{m}{\epsilon})$ gates each, total size of new circuit is 
        $O(m(log\frac{m}{\epsilon})^c)$
    \subsection{implementing measurement}
        we implement unitary operators $U=\sum \ketbra{i}{u_i}$

        \subsubsection{projective measurement}
        how to do projective measurement according to $\{P_i:i\in[k]\}$

        recall $P_i=\sum\ketbra{v_{ij}}{V_{ij}}$
         we implement basis change operator $U:=\sum\ketbra{i,j}{V_{ij}}$        

         \begin{itemize}
             \item apply U to get indices in register $A_1A_2$
             \item copy $A_1$ into ancilla B 
             \item Measure B 
             \item apply U* to $A_1A_2$
         \end{itemize}



    \subsection{efficiency, complexity classes}
         the complexity of implementing a measurement is captrued by the basis change operation

         if we can efficiently implement basis change operation. we can perform the measurement efficiently

         we say a family of cuicuit $\{C_n\}$ where it has n-qubit imput is 
         efficient if its size is $O(n^c)$ for a constant c


         the complexity class P consist of all family of boolean functions 
         $\{f_n|f_n:\{0,1\}^n\rightarrow\{0,1\}\}$ which has efficient deterministic 
         classical circuits 


         BPP: bouned error probablitistic polynomial time

         BQP: bouned error quantum polynomial time

         P: polynomial time 

    \subsection{simulating classcial algorithm}
        we use toffoli gate to simulate $\{And, Not\}$ given ancilla state $\ket{0},\ket{1}$

        clean simulation: we erase the contents of original output register AND the workspace 

        we do this by applying unitary transformation, and copying with cnot gate 

        complexity of clean simulation is efficient

    \subsection{simulating randomized algrithums, basic algorithm:blackbox, Deutrch-Jozza}
         for random circuit 

         Pr(C(x)=y)=pr(h(x,r)=y), where r is uniformly random over $\{0,1\}^k$,

        k is number of random bit 

        using quantum circuit, we apply $H^{\otimes k}$ to $\ket{0}$

        $\ket{\psi}=\frac{1}{\sqrt{2^k}}\sum_{r\in\{0,1\}}\ket{r}$

        $\ket{\varPsi}=U\ket{x}\ket{\psi}\ket{0}=\frac{1}{\sqrt{2^k}}
        \sum_r\ket{x}\ket{r}\ket{g(x,r)}\ket{h(x,r)}$

        $pr(output = y)=\frac{1}{2^k}\sum_r 1 (h(x,r)=y)=Pr(h(x,R)=y)$
         
        \begin{defn}
            blackbox or query model, we have a function $f \rightarrow \{0,1\}$
            and a circuit that computes it.

            classical: x-> f(x)

            quantum: 

            $\ket{x}\rightarrow \ket{x}$

            $\ket{b}\rightarrow \ket{b\ CNOT\ f(x)}$


            we call this circuit blackbox or oracle for x.
        \end{defn}


        \begin{defn}
            An assignment $a\in\{0,1\}^h$ of truth values to x satisfies 
            $\varphi$ if $\varphi(a)=1$
        \end{defn}

        a classical circuit C or a quantum one O for checking if a given assignment 
        satisfies $\varphi$ is a blackbox or oracle 

        given oracle for a function f, we wish to determine if f has some property 
        property like satisfiability

        the number of uses of the oracle in an algorithm, the number of queries,
         is called query complexity 

         \begin{thm}
             $H^{\otimes n}\ket{x}=\frac{1}{2^{n/2}}\otimes^n_{i=1}\left(
                 \sum _{y_i\in\{0,1\}}(-1)^{x_i-y_i}\ket{y_i}
             \right)
                 =\frac{1}{\sqrt{2^n}}\sum_y(-1)^{xy}\ket{y}$

            
            Also, $\frac{1}{2^n}\sum_x(-1)^{xy}$ = innerproduct of 
            $\frac{1}{\sqrt{2^n}}\sum_x\ket{x},\frac{1}{\sqrt{2^n}}\sum_x(-1)^{xy}\ket{x}$

            this is innerproduct of $H^{\otimes n}\ket{0^n},H^{\otimes n}\ket{y}$

         \end{thm}

         xy is scalar product $\sum x_iy_i$

         consider a circuit $\ket{x y}\rightarrow \ket{x (y\oplus f(x))}$
         

         if $f(x)=0,\ket{y}=\ket{-}, \text{second bit become} \\ U_f=\sqrt{1/2}*\ket{0\oplus f(x)} - \ket{1\oplus f(x)} =
         \sqrt{1/2}*(\ket{0}-\ket{1}) $
    
         if $f(x)=1,\ket{y}=\ket{-}, \text{second bit become} \\ U_f=\sqrt{1/2}*\ket{0\oplus f(x)} - \ket{1\oplus f(x)} =
         -\sqrt{1/2}*(\ket{0}-\ket{1}) $

         this is called phase kickback, where effect of function applying on second qubit becomes a $\pm $ on overall state 

         most generally, 
         $$U_f:(\alpha\ket{0}+\alpha_1\ket{1})(\ket{-})\rightarrow
         ((-1)^{f(0)}\alpha_0\ket{0}+(-1)^{f(1)}\alpha_1\ket{1})\ket{-}$$

        it can also be written as , 
         $$ (-1)^{f(0)}(\alpha_0\ket{0}+(-1)^{f(1)\oplus f(0)}\alpha_1\ket{1})\ket{-}$$

         we can determine $f(1)\oplus f(0)$ this way


         \subsection{simon problem}

         $Z_2^n$ forms a group under mod 2 $Z_2$

         $x+y=(x_1+y_1, x_2+y_2 ...)~~~0x=0~~~~ 1x=x$

         there is nonzero element $s\in Z_2^n$ such that  
         $f(x)=f(y)$ iff $y=x+s$ or $x=y$ 

         note $y=x+s, x=y+s, x+y=s$ are the same, we say f hides the string s.

         to find s we just need to find f(a)==f(b), b-a=s

         $U_f\left(\frac{1}{\sqrt{2}}\ket{00}+\frac{1}{\sqrt{2}}
         \ket{10}\right)=\frac{1}{\sqrt{2}}\ket{0 f(0)}+\frac{1}{\sqrt{2}}\ket{1 }\ket{0 +f(1)}$

         \begin{thm}[idea of simon algorithm]
             $$H^{\otimes n}\frac{1}{\sqrt{2}}(\ket{x}+\ket{x\oplus s})
             =\frac{1}{\sqrt{2^{n-1}}}\sum_{y\in S^{\perp}}
             (-1)^{xy}\ket{y}$$
         \end{thm}

    \section{phase estimation}
         unitary gate $e^{i\psi}$
         $$
         \begin{matrix}
             1 & 0\\
             0 & e^{i\psi}
         \end{matrix}$$
        
         $Pr(\ket{0})$ observed at $\ket{0}$ is $cos^2 (\psi/2)$
        at $\ket{1}$ is $sin^2 (\psi/2)$

        \subsection{textbook}
         \begin{thm}
             $$\frac{1}{\sqrt{2^n}}\su{2^n-1}{1}e^{2\pi i w y}\ket{y}==
             (\frac{\ket{0}+e^{2\pi i(2^{n-1}w)}\ket{1}}{\sqrt{2}})\otimes
             (\frac{\ket{0}+e^{2\pi i(2^{n-2}w)}\ket{1}}{\sqrt{2}})\otimes ...
             (\frac{\ket{0}+e^{2\pi i(2^{1}w)}\ket{1}}{\sqrt{2}})\otimes$$
             $$=(\frac{\ket{0}+e^{2\pi i(0.x_nx_{n+1}...) } } {\sqrt{2}})\otimes ...
             (\frac{\ket{0}+e^{2\pi i(0.x_1x_2...) } } {\sqrt{2}})\otimes$$
         \end{thm}

         \begin{thm}
             for phase estimation, $p(x)=\frac{sin^2(\pi(2^nw-x))}{2^{2n}sin^2(\pi (w-x/2^n))}$

             we also have lemma $\abs*{\theta M}<\pi /2$, then $\frac{1}{M^2}\frac{sin^2(M\theta)}{sin^2(\theta)}\geq 4/\pi^2$

             this applies to 2 nearest $2^nw$ to x

             with probability at least $1-\frac{1}{2(k-1)}$ phase estimation will ouptut at least 
             $2k $ closest integer multiple of $\frac{1}{2^n}$, $\abs*{w-\hat w}\leq \frac{k}{2^n}$ 

             final line: phase estimation will output one of the 2k closest integer multiples of $\frac{1}{2^n}$
             with probability at least $1-\frac{1}{2(k-1)}$
         \end{thm}

        \subsection{}
        more efficient way of estimating: say $\psi = 2\pi\theta$

        $\theta = \theta_1/2+\theta_2/2^2...+\theta_n/2^n$

        $$2^m\psi = 2\pi (2^m\theta) = 2\pi(2^{m-1}\theta_1+2^{m-2}\theta_2...+\theta_m+\frac{\theta_{m+1}}{2}+\frac{\theta_{m+2}}{2^2}...)
        =
        2\pi(0.\theta_{m+1}...\theta_1)$$
        
        \begin{itemize}
            \item we can learn $\theta_{m+1}$ by repeating experiment O(1) times, 
            \item apply the phase shift twice $2\psi \equiv 2\pi(0.\theta_2\theta_3...)$. and we learn $\theta_2$
            \item more generally, we can pllay the shift $2^m$ times and 
            $$2^m\psi\equiv 2\pi(0.\theta_{m+1}\theta_{m+2}...)$$
        \end{itemize}
        U be the unitary gate $e^{i\psi}$
        $$U^{2^i} H \ket{0} = \frac{1}{\sqrt{2}}(\ket{0}+e^{2\pi i(2^i\theta)}\ket{1})$$
        identify a string $y\in\{0,1\}^m$ with corresponding integer $\in\{0,1...,2^m-1\}$

        output state of $(U^{2^{m-1}}H\otimes U^{2^{m-2}}H\otimes  ...) \ket{000..}$
        $\ket{\psi_\theta} = \frac{1}{\sqrt{2^m}}\su{y=0}{2^n-1}e^{2\pi i \theta y}\ket{y}$
        

        suppose theta, which we are estimating, is $\theta = a/2^m$ for some $a\in X_{2^m}$

        then $e^{2\pi i a y/2^m} = w^{ay}$ where $w=e^{2\pi i/2^m}$ is a constant root of unity 

        define $\ket{X_x}=\frac{1}{\sqrt{2^m}}\su{2^m-1}{y=0}w^{xy}\ket{y}$

        then $\braket{x_a}{x_b}=0|a\neq b \text{ or } 1|a==b$ due to complex analysis

        \begin{defn}
            $\{\ket{X_x} 0\leq x < 2^m\}$ is then called the fourier basis.
        \end{defn}

        if $\theta = a/2^m$ for some a, then we can measure in fourier basis and determine $\theta$ exactly 

        we has Fourier transform operator 
        $$F_{2^m}=\su{2^{m-1}}{x=0} \ketbra*{X_x}{x}$$


        if $\theta$ is not integer multiple of $1/2^m$, of we measure in fourier basis ,
        pr(outcome = a) is $\abs*{\braket{X_a}{\psi_\theta}}^2$

    \subsection{eigenvalue estimation}
        V is unitary operator on $C^d$ s.t. we can apply $V^t$ geiven

        \begin{thm}
            note: when c*d is irrational, $1^{c*d}$ can take infinitely many value, and 
            $1^{c*d}\neq (1^c)^d$
        \end{thm}


    \section{error correction}
    \begin{defn}[Hamming code ]
        it encodes 1 bit into 3
    \end{defn}

    \begin{defn}[Hamming distance]
        the number of bits in which $x,y \in\{0,1\}^n$ differ
        
    \end{defn}

    \begin{defn}
        $(n,k,d)_2$ error correction code is a subset $C\in \{0,1\}^n$ of size $2^k$ st 
        $min\{\delta(x,y)\}:x y \in C, x\neq y=d$

        n is block length, k is message length, d is minimum distance of the code c

        the elements of the code is called codewords. ratio k/n is information rate

        we can recover x if $t\leq (d-1)/2$
    \end{defn}

    \begin{thm}
        for any $\epsilon\in [0,1/4)$, information rate $r<1-H(2\epsilon)$ 
        for all n large enough, there are $(n,k,d)$ error correction code with $k:=floor(rn)$
        and $d\geq 2\epsilon n +1$
    \end{thm}

    \begin{defn}
        a linear (n,k,d) code is [n,k,d] code

        it has a generator matrix
    \end{defn}

\section{shor code}
    \begin{thm}[9-qubit shor code]
        1 qubit into 3 for Z error 
        $\alpha \ket{0}+\beta\ket{1} \rightarrow \alpha\ket{+++}+\beta\ket{---}$

        then encode each of the three qubits into 3 qubits for X error 
        $\rightarrow \alpha\frac{1}{2\sqrt{2}}(\ket{000}+\ket{111})^{\otimes 3}+\beta\frac{1}{2\sqrt{2}}(\ket{000}-\ket{111})^{\otimes 3}$
    \end{thm}

    \begin{lem}
        if c is [n,k,d] code and $c^\perp$ is $[n,k_2,d_2]$ code with $d_1 d_2 \geq 2t+1$ 

        $\ket{\hat{\psi}} = \frac{1}{2^{k/2}}\sum_{x\in c}\ket{x}$ is a code that correct t errors
    \end{lem}

\section{encryption protocal}

    protocal pi 

    \begin{itemize}
        \item alice prepares k bell state $\ket{\psi}^k$ where $\ket{\psi}=\ket{00}+\ket{11}$ in register $A_1B_1$
        \item Alice encode state in $B_1$ using css code C, C has length n, encoded state is $B^\prime_1$
        \item alice prepare n bell state $\ket{\psi}^n$ in $A_2B_2$ with one qubit of each bell state in $A_2$  
        \item alice premute qubits in $B_1^\prime$ and $B_2$ uniformly random  Afterthis step, all communication is done classically
        \item bob acknowlodge recepit of 2n qubits 
        \item alice send premutation she used 
        \item bob inverts the premutation and obtain register $B_2^\prime B_2$
        \item alice selects  a unifromly random string $S\in\{0,1\}^n$ 
        \item alice measure ith qubit in $A_2$ in standard basis if $S_i=0$ in hardarmard basis if $S_i=1$, send mesasurement to bob
            
        \item alice send outcome of measurement to Bob 
        \item Bob measrue ith qubit in B2 in standard basis if $S_i=0$, or hadamard if 1 
        \item Bob compares outcome with those sent by alice 
        \item let $\delta_0$ be fraction st $S_i=0$ and different, $\delta_1$ simarly 
        \item if $\delta_0n \geq (\epsilon-v)*n/2$ ob informs alice, they output fail and stop. 
        \item  if both $\delta_0,\delta_1$ are small, bob use error correction to decode state in $B_1^\prime$ into $B_1$
        \item  alice and bob mearue k qubits $A_1 B_1$ instandard basis output pass and outcomes $K_A K_B$
    \end{itemize}

    suppose eve, easedropper apply $P\otimes V$ to qubits sent by alice and private register P, 
    suppose $P=\otimes_{i\in 1:2n}P_i$ where each $P_i$ is 1 X Z XZ

    suppose $\geq \epsilon 2n$ of $P_i\in \{X,XZ\}$ then output fail with prob $\geq 1-2exp(-cn)$

    proof: let $T\{i\in[2n]:P_i\in\{x,xz\}\}$

\end{document}




















