\documentclass[10pt]{article} 

\usepackage{fullpage}
\usepackage{bookmark}
\usepackage{amsmath}
\usepackage{amssymb}
\usepackage[dvipsnames]{xcolor}
\usepackage{hyperref} % for the URL
\usepackage[shortlabels]{enumitem}
\usepackage{mathtools}
\usepackage[most]{tcolorbox}
\usepackage[amsmath,standard,thmmarks]{ntheorem} 
\usepackage{physics}
\usepackage{pst-tree} % for the trees
\usepackage{verbatim} % for comments, for version control
\usepackage{tabu}
\usepackage{tikz}
\usepackage{float}

\lstnewenvironment{python}{
\lstset{frame=tb,
language=Python,
aboveskip=3mm,
belowskip=3mm,
showstringspaces=false,
columns=flexible,
basicstyle={\small\ttfamily},
numbers=none,
numberstyle=\tiny\color{Green},
keywordstyle=\color{Violet},
commentstyle=\color{Gray},
stringstyle=\color{Brown},
breaklines=true,
breakatwhitespace=true,
tabsize=2}
}
{}

\lstnewenvironment{cpp}{
\lstset{
backgroundcolor=\color{white!90!NavyBlue},   % choose the background color; you must add \usepackage{color} or \usepackage{xcolor}; should come as last argument
basicstyle={\scriptsize\ttfamily},        % the size of the fonts that are used for the code
breakatwhitespace=false,         % sets if automatic breaks should only happen at whitespace
breaklines=true,                 % sets automatic line breaking
captionpos=b,                    % sets the caption-position to bottom
commentstyle=\color{Gray},    % comment style
deletekeywords={...},            % if you want to delete keywords from the given language
escapeinside={\%*}{*)},          % if you want to add LaTeX within your code
extendedchars=true,              % lets you use non-ASCII characters; for 8-bits encodings only, does not work with UTF-8
% firstnumber=1000,                % start line enumeration with line 1000
frame=single,	                   % adds a frame around the code
keepspaces=true,                 % keeps spaces in text, useful for keeping indentation of code (possibly needs columns=flexible)
keywordstyle=\color{Cyan},       % keyword style
language=c++,                 % the language of the code
morekeywords={*,...},            % if you want to add more keywords to the set
% numbers=left,                    % where to put the line-numbers; possible values are (none, left, right)
% numbersep=5pt,                   % how far the line-numbers are from the code
% numberstyle=\tiny\color{Green}, % the style that is used for the line-numbers
rulecolor=\color{black},         % if not set, the frame-color may be changed on line-breaks within not-black text (e.g. comments (green here))
showspaces=false,                % show spaces everywhere adding particular underscores; it overrides 'showstringspaces'
showstringspaces=false,          % underline spaces within strings only
showtabs=false,                  % show tabs within strings adding particular underscores
stepnumber=2,                    % the step between two line-numbers. If it's 1, each line will be numbered
stringstyle=\color{GoldenRod},     % string literal style
tabsize=2,	                   % sets default tabsize to 2 spaces
title=\lstname}                   % show the filename of files included with \lstinputlisting; also try caption instead of title
}
{}
\newcommand{\su}[2]{\sum_{#1}^{#2}}

% floor, ceiling, set
\DeclarePairedDelimiter{\ceil}{\lceil}{\rceil}
\DeclarePairedDelimiter{\floor}{\lfloor}{\rfloor}
\DeclarePairedDelimiter{\set}{\lbrace}{\rbrace}
\DeclarePairedDelimiter{\iprod}{\langle}{\rangle}

\DeclareMathOperator{\Int}{int}
\DeclareMathOperator{\mean}{mean}

% commonly used sets
\newcommand{\R}{\mathbb{R}}
\newcommand{\N}{\mathbb{N}}
\newcommand{\Q}{\mathbb{Q}}
\renewcommand{\P}{\mathbb{P}}

\newcommand{\sset}{\subseteq}

\theoremstyle{break}
\theorembodyfont{\upshape}

\newtheorem{thm}{Theorem}[subsection]
\tcolorboxenvironment{thm}{
enhanced jigsaw,
colframe=Dandelion,
colback=White!90!Dandelion,
drop fuzzy shadow east,
rightrule=2mm,
sharp corners,
before skip=10pt,after skip=10pt
}

\newtheorem{cor}{Corollary}[thm]
\tcolorboxenvironment{cor}{
boxrule=0pt,
boxsep=0pt,
colback={White!90!RoyalPurple},
enhanced jigsaw,
borderline west={2pt}{0pt}{RoyalPurple},
sharp corners,
before skip=10pt,
after skip=10pt,
breakable
}

\newtheorem{lem}[thm]{Lemma}
\tcolorboxenvironment{lem}{
enhanced jigsaw,
colframe=Red,
colback={White!95!Red},
rightrule=2mm,
sharp corners,
before skip=10pt,after skip=10pt
}

\newtheorem{ex}[thm]{Example}
\tcolorboxenvironment{ex}{% from ntheorem
blanker,left=5mm,
sharp corners,
before skip=10pt,after skip=10pt,
borderline west={2pt}{0pt}{Gray}
}

\newtheorem*{pf}{Proof}
\tcolorboxenvironment{pf}{% from ntheorem
breakable,blanker,left=5mm,
sharp corners,
before skip=10pt,after skip=10pt,
borderline west={2pt}{0pt}{NavyBlue!80!white}
}

\newtheorem{defn}{Definition}[subsection]
\tcolorboxenvironment{defn}{
enhanced jigsaw,
colframe=Cerulean,
colback=White!90!Cerulean,
drop fuzzy shadow east,
rightrule=2mm,
sharp corners,
before skip=10pt,after skip=10pt
}

\newtheorem{prop}[thm]{Proposition}
\tcolorboxenvironment{prop}{
boxrule=0pt,
boxsep=0pt,
colback={White!90!Green},
enhanced jigsaw,
borderline west={2pt}{0pt}{Green},
sharp corners,
before skip=10pt,
after skip=10pt,
breakable
}

\setlength\parindent{0pt}
\setlength{\parskip}{2pt}


\begin{document}
\let\ref\Cref

\title{\bf{cs467}}
\date{\today}
\author{Austin Xia}

\maketitle
\newpage
\tableofcontents
\listoffigures
\listoftables
\newpage
\section{introduction and background}
The non-intuitive behavior of photon-mirror-beamsplitter system
results from features of quantum mechanics called \emph{superposition and interference}

the second beam aplitter has caused two paths (in superposition) to interfere,
resulting in cancelation of the 0 path

\section{linear albegra}
\begin{defn}[Hilbert space]
    finite-dimensional complex vector space H
\end{defn}
\subsection{Dual Vectors}
\begin{itemize}
    \item linearity in second argument 
    $$\langle v, \sum_i\lambda_iw_i\rangle =
    \sum_i\lambda_i\langle v, w_i \rangle $$
    \item conjugate-commutativity
    $$\langle v, w\rangle = \langle w, v\rangle^*$$
    \item nonnegativity 
    $$\langle v,v\rangle \geq 0$$
    with equality iff v=0
\end{itemize}
c* means complex conjugate of number c

$v\cdot w =\sum_i v_i^*w_i$

\begin{defn}[$H^*$]
    let H be a hilbert space, $H^*$ is the set of linear maps $H\rightarrow C$

We denote elements of $H^*$ by $\langle x|$, where $\langle X |$ has action
    $$\langle X|:|\psi\rangle\rightarrow\langle X | \psi \rangle \in C$$
\end{defn}
The set of maps $H^*$ is a complex vector space itself, is called the 
the \emph{dual vector space} associated with H.

The vector $\langle x|$ is called the dual of $|x\rangle$, 
it's derived from transposing $|x\rangle$ to row matrix and taking complex conjugate

Note dot product of complex vectors are 
$$\mathbf{A}\cdot\mathbf{B}=\sum_i a_i^*b_i$$

\begin{defn}[Euclidean norm of $|\psi\rangle$]
    norm of a vector $|\psi\rangle$ is $\norm{|\psi\rangle}\equiv \sqrt{\langle\psi|\psi\rangle}$
\end{defn}


\begin{defn}[Kronecker delta function$\delta_{i,j}$]
    it is 1 whenever i=j, 0 othervise
\end{defn}

\begin{thm}[dual basis]
    the set $\{\langle b_n|\}$ is an orthonormal basis for $H^*$ called the dual basis
\end{thm}

\subsection{operators}

outer product 
$$(|\psi\rangle\langle\varphi|)|\alpha\rangle =
|\psi\rangle(\langle\varphi|\alpha\rangle) = 
(\langle\varphi|\alpha\rangle)|\psi\rangle$$

outer project of a vector with itself defines a lienar operators $|\psi\rangle\langle\psi|$

\begin{defn}[orthogonal projector]
    it projects a vector in H to 1 dimensional subspace of H spanned by 
    $|\psi\rangle$. sunch an operator is called an orthogonal projector
\end{defn}

\begin{thm}
    let $B=\{|b_n\rangle\}$ be an orthonormal basis for 
    vector space $\mathbb{H}$, then every line operator T on H can be written as 
    $$T=\sum_{b_n, b_m \in B}T_{n,m}|b_n\rangle\langle b_m|$$
    where $T_{n,m}=\langle b_n | T | b_m \rangle$
\end{thm}

the set of all linear operators on vector space H forms a new complex vector space $L(H)$

vectors in L(H) are linear operators on H 
the basis vectors for L(H) are all possible outer products of pairs of basis vectors from B $\{|b_n\rangle\langle b_m|\}$

the action of T is then 
$$T:|\psi\rangle \rightarrow\sum_{b_n,b_m\in B}T_{n,m}\langle 
b_m|\psi\rangle|b_n\rangle$$

$T_{n,m}$ is matrix entry of T 

\begin{defn}[identity operator/resolution of the identity]
    for any orthonormal basis $B=\{|b_n\rangle\}$ identity operator is 
    $$1=\sum_{b_n\in B}|b_n\rangle\langle b_n |$$
\end{defn}

\begin{itemize}
    \item $H \rightarrow C  \in H^* $ corresponds to some vector $\langle \varphi^\prime|$
    \item adjoint of T, $T^\dagger$ sends $|\varphi\rangle \rightarrow |\varphi^\prime \rangle$
\end{itemize}

\begin{defn}
    adjoint of T, $T^\dagger$ is linear operator on $H^*$ that satisfies 
    $$(\langle\psi|T^\dagger|\varphi\rangle)^*=(\langle\varphi|T|\psi\rangle)~~~\forall|\psi\rangle, |\varphi\rangle \in H$$
    
\end{defn}

$T^\dagger$ is just complex conjugate transpose of T,
also called Hermitean conjugate, adjoint of T

\begin{defn}[unitary]
    U is unitary if $U^\dagger=U^{-1}$
\end{defn}

\begin{defn}[Hermitean]
    an operator T in Hilbert space H is called Hermitean (self adjoint) if 
    $$T^\dagger=T$$
\end{defn}

\begin{defn}[projector, orthogonal projecter]
    A projector on vector space H is a linear operator P that satisfies 
    $P^2=P$, an orthogonal projector also satisfies 
    $P^\dagger=P$
\end{defn}

\begin{thm}
    if $T==T^\dagger~~T|\psi\rangle==\lambda|\psi\rangle$, then 
    $\lambda \in R$
\end{thm}

\begin{defn}[trace]
    $Tr(A)=\sum_{b_n}\langle b_n|A|b_n\rangle$
    where $\{|b_n\rangle\}$ is any orthonormal basis
\end{defn}

\subsection{spectral Theorem}

\begin{defn}[normal]
    A is normal operator if 
    $$AA^\dagger=A^\dagger A$$
    both unitary and hermitean operators are normal
\end{defn}

\begin{thm}[spectral Theorem]
    for every normal operator T acting on a finite-dimentional Hilbert space H,
    there is an orthonormal basis of H consisting of eigenvectors $|T_i\rangle$ of T

    We refer to T written in its own eigenbasis as the spectral decomposition of T.

    The set of eigenvalues of T is called the sepctrum of T
\end{thm}

$T=\sum_iT_i|T_i\rangle\langle T_i|$ where $T_i$ are eigen values, $|T_i\rangle$ are eigenvectors

another way of saying the spectral theorem is 
\begin{thm}
    For every finite dimensional normal matrix T there is a unitary matrix 
    P such that $T=PDP^\dagger$ where D is a diagonal matrix

    diagonal entries of D are eigenvalues of T. columns of P encode eigenvectors of T
\end{thm}

\subsection{Functions of Operators}

with spectral theorem we can write any normal operator T to 
$$T=\sum_iT_i|T_i\rangle\langle T_i|$$

Note each $|T_i\rangle\langle T_i|$ is a projector


Talor series for any function f acting on an operator T will have 
$$f(T)=\sum_ma_mT^m$$
If T is written in diagonal form,
then$$f(T)=\sum_if(T_i)|T_i\rangle\langle T_i|$$

\subsection{Tensor Products}
\begin{itemize}
    \item For any $c\in\mathbf{C},|\psi_1\rangle\in H_1, |\psi_2\rangle\in H_2$
    $$c(|\psi_1\rangle\otimes|\psi_2\rangle)=
    (c|\psi_1\rangle)\otimes|\psi_2)=
    |\psi_1\rangle\otimes(c|\psi_2\rangle)$$
    \item for any $|\psi_1\rangle$, $|\varphi_1\rangle\in H_1$, $|\psi_2\rangle\in H_2$
    $$(|\psi_1\rangle+|\varphi_1\rangle)\otimes|\psi_2\rangle=
    |\psi_1\rangle\otimes |\psi_2\rangle+|\varphi_1\rangle\otimes|\psi_2\rangle$$
    \item for any $|\psi_1\rangle$, $|\varphi_1\rangle\in H_1$, $|\psi_2\rangle\in H_2$
    $$|\psi_1\rangle\otimes(|\psi_2\rangle+|\varphi_2\rangle)=
    |\psi_1\rangle\otimes |\psi_2\rangle+|\psi_1\rangle\otimes|\varphi_2\rangle$$
\end{itemize}

$$(A\otimes B)(|\psi_1\otimes|\psi_2\rangle)\equiv A|\psi_1\rangle \otimes B |\psi_2\rangle$$


\section{Qubits and the framework of quantum mechanics}
Quantum information is the result of reformulating information theory 
in this quantum framework

\subsection{State of a Quantum System}
the light-splitter example is an example of a 2-state quantum system:

a photon that follow one of two paths, which we identify by 
$\begin{pmatrix}
   1\\
   0 
\end{pmatrix}$
and
$\begin{pmatrix}
   0\\
   1 
\end{pmatrix}$

and noted a path state of the photo can be described as 
$\begin{pmatrix}
   \alpha_0\\
   \alpha_1
\end{pmatrix}$
with $|\alpha_0|^2+|\alpha_1|^2=1$

\begin{defn}[State Space Postulate]
    the state of a system is described by a unit vector in a Hilbert Space $\mathbb{H}$
\end{defn}

$\mathbb{H}$ can be infinite dimensional.

In practice, we cannot distinguish a continous state from a discrete state having small spacing

we label one basis vector with $|0>$, one with $|1>$, they are orthogonal to each other

we represent general state by $\alpha_0 |0>+\alpha_1 |1>$
with $|\alpha_0|^2+|\alpha_1|^2=1$

$\alpha_0$ and $\alpha_1$ are complex coefficient, often called \emph{amplitude} of basis state $|0>$ and $|1>$

amplitude $\alpha$ can be decomposed uniquely as a product $e^{i\theta}|\alpha|$
where $|\alpha|$ is non-negative corresponding to magnitude of $\alpha$,
$e^{i\theta}$ has norm 1.

value $\theta$ is \emph{phase}, $e^{i\theta}$ is \emph{phase vector}

the state is described by unit vector means $|\alpha_0|^2+|\alpha_1|^2=1$

this is called \emph{normalization constriant}

vector $e^{i\theta}|\phi>$ is equivalent to the state described by $|\phi>$

example of $\phi$ would be $|0> + |1>$

question: is this example not considering it has 100\% appearing in 0 and 1 state

On the other hand, relative phase factors between 2 orthogonal states in superposition
are physically significant, and the state described by the vector

$|0>+|1>$ is physically different from $|0>+e^{i\theta}|1>$

\begin{thm}[State SpacePostulate]
    we can describe the most general state $|\phi>$ of a single qubit by a vector of form 
    $$|\phi>=cos(\theta/2)|0> + e^{i*k}sin(\theta/2)|1>$$
\end{thm}


on a classical computer, a \emph{classical bit} could be represented by a 0/1,

there is also probablitistic classical bit
$\begin{pmatrix}
    p_0\\p_1
\end{pmatrix}$

we can represent two probabilities by 2-dimensional unit vector 

now we go back to quantum bit, which is described by a complex unit vector 
$|\phi>$ in 2 dimentional Hilbert space. Up to a (physically insignificant)
flobal phase factor, such a vector can be written in the form 
    $$|\phi>=cos(\theta/2)|0> + e^{i*k}sin(\theta/2)|1>$$

this state vector is often depicted as a point on a 3-dimentional sphere, the \emph{Bloch Sphere}

points on the sphere can be expressed in Cartesian corrdinates as 

$$(x,y,z)=(sin\theta cos\phi, sin\theta sin\phi, cos\theta)$$

\subsection{Time Evolution of a Closed System}

\begin{thm}[Evolution Postulate]
    The time-Evolution of state of a \emph{closed} quantum system 
    is described by a unitary operator. that is for any evolution of 
    the closed system there exsist a unitary operator U s.t. if initial state 
    is $|\psi_i>$ then after evolution the state will be 
    $$|\psi_2> = U |\psi_1>$$
\end{thm}

\subsection{Composite Systems}

How to described a closed system of n qubits. how it envolves and what happens when we measure it. 

we treat it as a composition of subsystems

\begin{thm}[Composition of Systems Postulate]
    When two physical systems are treated as one combined system.
    the state space of combined physical system is product space $H_1\otimes H_2$ of 
    the state space $H_1, H_2$ of the component subsystems.

    If first system is in state $|\psi_1>$, second system in state $|\psi_2>$,
    the state of combined system is $$|\psi_1>\otimes |\psi_2>$$
\end{thm}

We write the joint state like $|\psi_1\rangle|\otimes\psi_2\rangle$ or $|\psi_1\rangle|\psi_2\rangle$

If 2 qubits are allowed to intreact, the closed system includes both qubits together, 
it might not be possible to write the state in product form. 
Then we say the qubits are \emph{entangled}

The state of composite system is a vector in 4-dimensional tensor product space of 
2 constituent qubits. The 4-dimensional state vectors formed by tensor product of 
the 2 2-dimensional state vectors form a sparce subset of all 4-dimensional state vectors. 

Most 2-qubits states are entangled.

A state is tangled if the equation has no solution of $\alpha$ and $\beta$
$$|\psi\rangle=(\alpha_0|0\rangle+\alpha_1|1\rangle)(\beta_0|0\rangle+\beta_1|1\rangle)$$

for example $$|\psi\rangle=\frac{1}{\sqrt{2}}|0\rangle|0\rangle
+\frac{1}{\sqrt{2}}|1\rangle|1\rangle$$

This is EPR pair. If we use X to first qubit, Y to second qubit,

the system $|\psi_1\rangle\otimes|\psi_2\rangle$ is mapped to 
$$X|\psi_1\rangle\otimes I|\psi_2\rangle=(X\otimes I)(|\psi_1\rangle\otimes|\psi_2\rangle)$$

\subsection{Measurement}
state of a single-qubit system is represented as a vector in Hilbert space

The evolution of state of a system during a Measurement is not unitary 

if the state $\sum_i\alpha_i|i>$ is provided as input.
it will output i with probability $|\alpha_i|^2$ and leave the system in state $|i>$

\begin{thm}
    For a given orthonormal basis $B=\{|\varphi_i \}$ of a state space $H_A$
    for a system A, it's possible to perform a Von Neumann measurement on system $H_A$
    with respect to basis B that, given a state 
    $$|\psi\rangle = \sum_i\alpha|\varphi_i\rangle$$
    outputs label i with probablity $|\alpha_i|^2$ and leaves the system in state $|\varphi_i\rangle$

    For state $|\psi\rangle=\sum_i\alpha_i|\varphi_i\rangle$, note that 
    $\alpha_i=\langle\varphi_i|\psi\rangle$ and thus 
    $$|\alpha_i|^2=\alpha_i*\alpha_i=\langle\psi|\varphi_i\rangle\langle\varphi_i|\psi\rangle$$
\end{thm}


One slight generalization of Von Neumann measurements:

A Von Neumann Measurement is special kind of projective Measurement

Recall orthogonal projection is operator P with 

Note: $A^\dagger$ stands for congugate transpose of A
$$P^2=P, P^\dagger = P$$

For any decomposition of identy operator $I=\sum_iP_i$
into orthogonal projectors $P_i$, there exisits projective Measurement
that outputs i with probablity $p(i)=\langle \psi|P_i|\psi\rangle$
and leaves the system in renormalized state $\frac{P_i|\psi\rangle}{\sqrt{p(i)}}$.

In other words, this measurement projects the input state $|\psi\rangle$ into 
one of the orthogonal subspaces corresponding to the projection operators 
$P_i$, with probablity of sqaure of amplitude of component of $|\psi|\rangle$ in that subspace


Von Neumann measurement is special case of a projective measurement where all projectors $P_i$ has rank one 
(in other words, are of $|\psi_i\rangle\langle\psi_i|$)

The simplest example of a Von Neumann measurement is a complete measurement in the 
computational basis. This can be viewed as following decompositon 
$$I=\sum_{i\in\{0,1\}^n}P_i$$
where $P_i=|i\rangle\langle i|$

projective measurements are often described as \emph{observable}
An observable is a Hermitean operator M acting on state space of the system 
it has decomposition
$$M=\sum_im_iP_i$$
where $P_i$ is orthogonal projector on eigenspace of $M$ with real eigenvalue $m_i$

Measuring the observable corresponds to performing a projective measurement 
with respect to the decomposition $I=\sum_iP_i$ where the measurement outcome 
i corresponds to eigenvalue $m_i$

$$\norm*{|\psi\rangle}\equiv ... $$


\section{circuit model of computation}
uniform families of reversible circuits: model of computation

 Circuits are networks composed of wires that carry bit values to gates that 
 perform elementary operations on the bits

 acyclic, bits move through circuit in a linear fashion, wires never feed back to a prior location in the circuit. 

 A circuit has n wires. input -> enter at most one gate at one time -> output

 A family of circuits is a set of circuits $\{C_n|n\in Z^+\}$

 \begin{defn}
     a set of gates is universal if for any f, n,m $f:\{0,1\}^n\rightarrow \{0,1\}^m$

     a circuit can be constructed for f using only gates from that set
 \end{defn}

 for circuits model, measuring complexity,

 we can measure the number of gates/depth of the circuit.

 the time slices (different from gate as concurrency)

 bits, number of lines

\section{blackbox model}
    the input is provided by a blackbox or oracle. for aceessing information 
    about an unkonwn string 

\end{document}
